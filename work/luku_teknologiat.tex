	\section{Sisätilapaikannuksen teknologiat}
Tässä luvussa tarkastellaan erilaisia teknologioita sisätilapaikannuksen toteuttamista varten. Luvussa tarkastellaan jokaisen teknologian hyviä ja hunoja puolia sisätilapaikannuksen näkökulmasta. Lopuksi tarkastelemme millainen tulevaisuus sisätilapaikannus teknologioilla voisi olla, onko tulevaisuus fragmentoitunut teknologioiden osalta, eli onko olemassa useita teknologioita samaan aikaan käytössä vai onko olemassa yksi vallitseva teknologia. Luvun lopussa on vielä yhteenveto luvun aiheesta.
\subsection{WLAN}
%teknologian esittely
WLAN(Wireless Local Area Network) on selvästi yleisimmin standardin IEEE 802.11 mukainen langaton lähiverkko\cite{A}. WLAN:in kantama on 50 - 100m ja WLAN tukiasemia löytyy nykyään lähes joka puolelta, niin kotitalouksista kuin julkisista tiloistakin.
%paikannus tekniikat

WLAN paikannus tekniikoita on kahdenlaisia. Toiset perustuvat trilateration menetelmään eli mittaamalla signaalin vahvuutta vähintään kolmeen tunnettuun WLAN tukiasemaan. Toiset perustuvat sormenjälki menetelmään, jossa WLAN signaalin vahvuudesta referenssi pisteissä tehdään tietokanta, siten että signaalien vahvuudet ja referenssipisteen sijainti tallennetaan tietokantaan.\cite{G}
%trilateration
Trilateration menetelmä on yksinkertainen. Paikannukseen tarvitaan vähintään kolme tukiasemaa, joihin mitataan etäisyys mittaamalla signaalin vahvuus kuhunkin tukiasemaan ja muuttamalla tämä arvo etäisyydeksi. Kolmen tukiaseman etäisyyden perusteella voidaan määrittää sijainti, kun tukiasemien sijainnit tunnetaan. Sijainnin määrittämiseen voidaan käyttää esimerkiksi pienin neliösumma menetelmää.\cite{G}Tällä menetelmällä voidaan päästä muutaman metrin tarkkuuteen\cite{G} mutta signaalin vahvuus ei aina vastaa suoraan etäisyyttä ympäristön häiriöiden takia ja tarkkuus voi jäädä kymmeniin metreihin \cite{A}. Microsoftin tutkimusryhmän kehittämä RADAR\cite{H} järjestelmä perustuu tähän tekniikkaan ja sillä päästiin muutaman metrin tarkkuuteen 50 prosentin todennäköisyydellä.
%fingerprinting

Sormenjälki menetelmässä on kaksi vaihetta\cite{G}. Ensin rakennetaan tietokanta, jossa on referenssi pisteissä kuuluvien WLAN tukiasemien signaalivahvuudet. Seuraavassa vaiheessa sijainti selvitetään mittaamalla signaalivahvuudet ja vertaamalla niitä tietokantaan. Tietokantaan voidaan tallentaa signaalivahvuuksien keskiarvo jolloin puhutaan deterministisestä menetelmästä tai signaalivahvuuksista voidaan määrittää todennäköisyysjakauma, jolloin päästään suurempaan tarkkuuteen mutta myös tietokannan koko kasvaa suureksi.\cite{G}

WLAN paikannuksen heikkoutena on signaalin voimakkuuden suuri vaihtelu ympäristön vaikutusten takia\cite{B,G}, muun muassa ihmisen vartalon asento ja suunta, ovet, seinät ja antennien asennot vaikuttavat voimakkuuteen. Erityisesti sormenjälki menetelmän heikkoutena ovat muuttuvat ympäristöt, kuten uudet seinät ja sermit jotka heikentävät WLAN signaalin etenemistä, jolloin tietokantaan täytyy päivittää uudet signaalivahvuus tiedot.

WLAN paikannuksen vahvuuksia ovat sen yleisyys ja nykyisten jo olemassa olevien laitteiden helppo muuntaminen tähän käyttöön\cite{G}. WLAN paikannuksen infrastruktuuri on myöskin jo olemassa monilla julkisilla paikoilla ja on siten vähentävät paikannuksen kustannuksia\cite{A,B}.
%hyvät ja huonot puolet
%+joka puolella
%+valmis infra
%-ei tarkoitettu tähän
\subsection{Bluetooth Low Energy}%BBBBBBBBBBBBBBBBB
%iBeacon
%teknologian esittely
Bluetooth Low Energy (BLE) toimii samalla 2.4GHz taajuus alueella ja sen kantama on samaa luokkaa kuin WLAN teknologialla\cite{BLE,B}. BLE standardin mukaiset BLE majakat lähettävät lyhyitä viestejä laitteille, joiden avulla nämä laiteet voi signaalin vahvuuden mukaan tunnistaa läheisyyden majakkaan. BLE ominaisuuksiin kuuluu myös pieni virrankulutus, varsinkin verrattuna WLAN teknologiaan\cite{ibeacon}.

%paikannus tekniikat
BLE majakoiden avulla paikannus perustuu majakoiden tunnettuihin sijainteihin. Majakat mainostavat omaa tunnistenumeroaan alueella oleville laitteille. Laitteet sitten lähettävät saamansa tiedon palvelimelle, yleensä WLAN:in avulla. Palvelin lähettää laitteelle laitteen sijainnin. Tämänlaista järjestelyä on käytetty artikkelissa\cite{ibeacon} puhutussa sairaalassa.

BLE majakoiden avulla, sisätilapaikannus voidaan myös toteuttaa samoin kuin WLAN teknologialla\cite{BLE}. Trilateration menetelmä ja sormenjälki paikannus menetlemä onnistuvat kumpikin mutta BLE kärsii samoista ongelmista kuin WLAN signaalin vaimentumisen ja heijastumien takia. BLE kärsii myös suuremmasta signaalinvahuuden vaihteluista kuin samalla taajuusalueella toimiva WLAN, johtuen pienemmästä kaistanleveydestä\cite{BLE}. Tutkimuksessa\cite{BLE} kuitenkin todetaan, että BLE sormenjälki paikannuksella voidaan saada parempia tuloksia kuin nykyisten kaltaisilla WLAN verkoilla. Tämä kuitenkin vaatii useita mittauksia ja kasvattaa sormenjälkitietokannan kokoa ja on työlästä eikä täten aina kannattavaa.
%hyvät ja huonot puolet
BLE teknologian houkuttelevuus perustuu halpaan hintaan ja BLE laitteiden yleisyyteen ja saatavuuteen. Mikä tahansa BLE tukeva laite voidaan asettaa majakaksi\cite{BLE}. Käytännössä kaikissa uudemmissa puhelimissa on Bluetooth tuki, joten BLE teknologia on hyvä vaihtoehto sisätilapaikannus sovelluksien kannalta. BLE teknologian etuihin kuuluu myös yksisuuntainen tiedon siirto. Yksisuuntaisella tarkoitetaan sitä, että majakat lähettävät signaalia mutteivät vastaanota ja sijainnin määrittäminen jää laitteen vastuulle. Tällä tavalla ei haluttu seuranta, ei ole mahdollista. Applen kehittämä iBeacon protokolla perustuu BLE teknologiaan\cite{bluesentinel}. Applen kiinnostuksella BLE teknologiaan on suuri myötävaikutus sen yleistymiseen.

BLE majakoilla sellaisenaan ei saada täydellisen tarkkaa sijaintia, vaan tieto siitä, millä alueella seurattava laite sijaitsee. Monissa sovelluksissa tämä on riittävä tieto ja yhdistettynä pieneen virrankulutukseen ja yleisyyteen, BLE on varteenotettava vaihtoehto sisätilapaikannuksen toteuttamiseksi. 
%BBBBBBBBBBBBBBBBBBBBBBBBBBBBBBBBBBBB
\subsection{RFID}
RFID(radio taajuinen tunnistus) järjestelmään kuuluu useita komponentteja, mukaan lukien useita RFID lukijoita ja tunnisteita. RFID lukija voi lukea tunnisteiden lähettämää informaatiota. Tunnisteet voidaan jakaa kahteen kategoriaan, aktiivisiin ja passiivisiin. Passiiviset tunnisteet toimivat ilman virtalähdettä ja saavat tarvittavan energian informaation lähettämiseen lukijalta. Aktiiviset tunnisteet vaativat toimiakseen oman virtalähteen, joka on usein pieni paristo. Niillä on myös suurempi kantama kuin passiivisilla.\cite{E}

Sisätilapaikannus käyttäen RFID teknologiaa voidaan jakaa kahteen sen mukaan ovatko lukijat vai tunnisteet kiinteitä. Kiinteällä tarkoitetaan sitä, että tunniste tai lukija on paikoillaan tunnetussa sijainnissa. 

Tutkimuksessa \cite{E} on käytetty kiinteitä lukijoita. Tällä tekniikalla seurattavassa kohteessa on tunniste ja seurattavalle alueelle on sijoitettu useita lukijoita. Jokainen lukija on asetettu sopivalle tehotasolle siten, että sen kantama tunnetaan. Näin alue jaetaan pieniin osa-alueisiin, jotka voidaan tunnistaa sen perusteella mitkä lukijat pystyvät lukemaan tunnisteen tältä alueelta. Monet asiat vaikuttavat lukioiden kantamiin, siksi virheen mahdollisuus on usein suuri.
Samassa tutkimuksessa\cite{E} esiteltiin myös paranneltu tekniikka nimeltä LANDMARC ( Location Identification based on Dynamic Active RFID Calibration). Tässä tekniikassa alueella on lukijoiden lisäksi viitetunnisteita tunnetuissa sijainneissa. Niiden avulla saadaan tarkempi tulos kun seurattavan tunnisteen etäisyyttä verrataan viitetunnisteiden tunnettuihin etäisyyksiin. Etäisyys määritetään muuttamalla lukijan tehotasoa ja täten kantamaa.

Toinen tapa on käyttää pelkästään kiinteitä tunnisteita ja lukijan sijainti määritetään niiden perusteella. Näin on tehty muun muassa tässä tutkimuksessa\cite{F}. Tutkimuksessa käytettiin passiivisia tunnisteita lukijan reitin varrella ja sijainti määritettiin mittaamalla tunnisteen takaisin lähettämä teho, joka riippuu pääasiassa lukijan ja tunnisteen välisestä etäisyydestä\cite{F}.

RFID teknologiaa hyödyntämällä voidaan päästä erittäin tarkkoihinkin tuloksiin, tutkimuksessa \cite{E} päästiin noin metrin tarkkuuteen ja tutkimuksessa \cite{F} jopa 0.1 metrin tarkkuuteen. RFID teknologian vahvuus on selvästi hyvä tarkkuus, mutta tähän tarkkuuteen pääseminen vaatii paljon mahdollisesti kallista infrastruktuuria. Varsinkin tapauksessa\cite{E}, jossa käytetään kiinteitä lukijoita, infrastruktuuri tulisi kalliiksi varsinkin suurilla alueilla. Tutkimuksen\cite{F} mukainen järjestely ei myöskään ole kovinkaan toimiva suurilla alueilla, sillä hyvän tarkkuuden saavuttamiseksi, koko alue tulisi peittää tiheällä tunniste verkolla. Vaikka passiiviset tunnisteet ovatkin halpoja\cite{F}, tiheän verkon asentaminen ei ole mielekästä.

\subsection{VLC}
VLC(Visible Light Communication) eli näkyvän valon kommunikaatiolla tarkoitetaan nimensä mukaisesti tiedonsiirtoa, joka tapahtuu näkyvän valon avulla\cite{VLCA}. Lähettämiseen käytetään useimmiten LED valoja niiden nopeuden, kestävyyden ja energiatehokkuuden takia. Vastaanottaminen tapahtuu useimmiten puhelimen kameralla.

Tutkimuksessa\cite{VLCA} paikannus on toteutettu jakamalla alue ruudukoksi siten että jokaisen ruudun kulmassa on LED valo. Sijaintitieto välitetään binaarikoodina vastaanottimelle ja häiriöiden välttämiseksi kulmissa olevat LED valot on ajoitettu lähettämään sijaintitietoaan eri aikoihin(TDM, Time Division Multiplexing), jolloin vain yksi kerrallaan lähettää, eivätkä signaalit mene sekaisin. Muina aikoina kuin lähettäessä valo on päällä normaalisti tarjoamassa valaistusta ympäristöönsä. 
Tarkempi sijainti ruudussa saadaan valojen intensiteetistä. Kaikilla LED valoilla on puoliteho kulma 60\degree ja valo jakautuu useimmiten ensimmäisen asteen Lambertin kuvion(Lambertian pattern) mukaan. Näillä tiedoilla valon intensiteettiin voidaan muuttaa kulmaksi jossa valo on vastaanottimeen nähden. Kun kulmat ovat selvillä kaikkiin ruudun 4 nurkkaan, voidaan sijainti laskea.\cite{VLCA}
Tällä tekniikalla on päästy erittäin tarkkoihin tuloksiin, jopa alle millimetrin tarkkuuteen vähä häiriöisessä ympäristössä ja muutamiin kymmeniin millimetreihin häiriöisessä ympäristössä.

Näkyvän valon kommunikaation etuja ovat hinta, tarkkuus ja energia tehokkuus. LED valot yleistyvät joka puolella joka tapauksessa energiatehokkuutensa ansiosta, joten niiden käyttö sisätilapaikannukseen on selvä etu, kun uutta ja kallista erillistä infrastruktuuria ei tarvita. VLC teknologialla päästään myös selvästi tarkempiin tuloksiin kuin millään edellä mainitulla teknologialla.
VLC teknologian ongelmaksi muodostuu vastaanottimen asennon tärkeys. Toimiakseen hyvin vastaanottimen täytyy olla melko tarkasti vaakatasossa, eikä tämä ole kaikissa sovelluksissa mahdollista tai mielekästä. Lisäksi vastaanottimen ollessa kädessä, ihminen saattaa peittää osan lähettimistä, jolloin tarkkuus kärsii.
%visible light communication

%teknologian esittely
%paikannus tekniikat
%hyvät ja huonot puolet

\subsection{Liikeanturit}
%gyro,accelometer,compass
Sisätilapaikannus perustuen liikeantureihin poikkeaa oleellisesti kaikista edellä mainituista teknologioista ja tekniikoista. Edellä mainitut teknologiat perustuvat jonkinlaiseen seurattavan laitteen ulkopuoliseen infrastruktuuriin sijainnin selvittämiseksi. Liikeantureista kuten gyroskoopeista, kiihtyvyys antureista ja kompasseista koostuva IMU(Inertial Measurement Unit) eli inertiamittausyksikön toiminta perustuu liikkeen seurantaan. Laitteella täytyy olla yksi tarkka tunnettu sijainti, joka voidaan saada esimerkiksi GPS tai jollain tässä työssä mainitulla teknologialla.\cite{IMU} Tämän jälkeen IMU alkaa laskea laitteen sijaintia integroimalla nopeuden, kiihtyvyyden ja suunnan muutoksia. 

Paikannuksen toiminta edellyttää hyvin tarkkoja mittaustuloksia IMU:lta. Pienetkin virheet kumuloituvat nopeasti useiksi metreiksi ja lopulta kilometreiksi. Monissa uusissa puhelimissa on kaikki tämän teknologian vaatimat anturit mutta niiden tarkkuus on aivan liian heikko toimiakseen. Tutkimuksessa\cite{IMU} käytetty HG1700 on erittäin tarkka taktisen tason IMU jota on käytetty muun muassa ohjuksissa\cite{hg1700}. Tällä järjestelyllä päästiin silti vain kohtuulliseen tarkkuuteen ja virhettä oli enimmillään 5 m 40 minuutin testin aikana\cite{IMU}. 

Liikeanturien käyttö sisätilapaikannuksessa on mahdollista mutta hyödyllisiin tuloksiin päästäkseen on käytettävä erittäin tarkkoja ja kalliita laitteita. Tämä ei ole mahdollista kuluttaja sovelluksissa mutta liikeantureita voidaan käyttää sijainnin arvioimiseen, kun muita tekniikoita ei ole hetkellisesti käytössä.
%teknologian esittely
%paikannus tekniikat
%hyvät ja huonot puolet
\subsection{Magneettianturi}
%indoorAtlas
Perinteisesti magneettiantureita on käytetty kompasseina maanmagneettikentän suuntaa mittaamalla. Ulkotiloissa tämä toimiikin hyvin ja kompassi osoittaa pohjoiseen. Sisätiloissa kompassi saattaa osoittaa aivan väärään suuntaan ja tämä suunta voi vaihdella suurestikin. Suomalainen yritys, IndoorAtlas Ltd. on kehittänyt sisätilapaikannus teknologian, joka perustuu juurikin näihin maan magneettikentän epämuodostumiin\cite{atlas}. Epämuodostumia aiheuttaa pääasiassa rakennusten teräsrakenteet, mutta myös sähköjohdot, putket ja elektroniset laitteet luovat epämuodostumia maan magneettikenttään\cite{magneetti}.
%teknologian esittely

%paikannus tekniikat
IndoorAtlasin kehittämä ja patentoima paikannusteknologia perustuu samanlaiseen sormenjälkitekniikkaan, kuten edellä on esitelty. Alueesta täytyy siis ensin tehdä tietokanta johon tallennetaan referenssipisteiden magneettikentän suunta ja voimakkuus\cite{atlas}. Tämän jälkeen paikannus tapahtuu vertailemalla jonkin pisteen magneettikentän voimakkuutta ja suuntaa tietokantaan tallennettuihin pisteisiin. Ensimmäisen sijainnin löytäminen vaatii laitteen liikkumista, jotta saadaan selville usampi kuin yksi piste\cite{atlas}. Tämä johtuu siitä, että monessa pisteessä voi olla samankaltainen magneetti kenttä, mutta liikuttaessa selvitetään usean vierekkäisen pisteen magneettikenttä jolloin paikannus voidaan toteuttaa paremmalla varmuudella.
IndoorAtlas lupaa teknologialle 1-2 metrin tarkkuutta\cite{atlas}, mutta tutkimuksessa\cite{magneetti} on päästy jopa desimetrien tarkkuuteen.
%hyvät ja huonot puolet

Tämän teknologian selviä vahvuuksia on toiminta ilman erillistä infrastruktuuria, maan magneettikentän esiintyminen joka puolella maailmaa ja sen vakaus\cite{magneetti}. Teknologia kärsii kuitenkin sijaintitiedon vähyydestä, sillä tietokannan tekemiseen saatetaan esimerkiksi käyttää vain magneettikentän suuntaa jos mittava laite ei esimerkiksi pysty antamaan tarkempaa tietoa\cite{magneetti}. Tällöin tietokannassa on paljon samanlaisia sijainti ja magneettikenttä pareja jolloin oikean sijainnin löytäminen on haasteellista. Lisäksi teknologia kärsii samasta ongelmasta, kuin kaikki muutkin sormenjälkimenetelmään perustuvat teknologiat. Ympäristön muuttuessa, esimerkiksi kaupassa hyllyjen paikkaa vaihdetaan, tietokanta joudutaan päivittämään muuttuneen ympäristön osalta. Jos muutoksia tapahtuu usein, tietokannan päivittäminen ilman automaatioita on työlästä.

\subsection{Sisätilapaikannus teknologioiden tulevaisuus}
%fragmentoitunut vs yksi vallitseva
Tällä hetkellä ei ole olemassa yhtä standardi sisätilapaikannus teknologiaa, kuten voi jo tämän luvun sisällöstäkin päätellä. Yksi tämän työn tavoitteista on tehdä arvio sisätilapaikannus teknologioiden tulevaisuudesta, eli mikä tai mitkä teknologiat yleistyvät ja onko tulevaisuudessa käytössä rinnakkaisesti useita teknologioita rinnakkain vai onko valloilla vain yksi teknologia jota käytetään kaikissa sisätilapaikannus sovelluksissa. Tämän alaluvun tarkoitus on vastata tähän kysymykseen.

Yhden teknologian tulevaisuutta puoltaa se, että kun ensimmäinen läpimurto sovellus luodaan, sen suosion myötä myös sen käyttämä teknologia lisää näkyvyyttään ja suosiotaan. Näin ollen sisätilapaikannus sovelluksia luovat tahot todennäköisimin tulevat käyttämään tätä teknologiaa, ja syntyneellä kierteellä kyseisen teknologian suosio kasvaa edelleen. Toisaalta monen teknologian tulevaisuutta puoltaa se, että eri tiloissa ja sovelluksissa toimii parhaiten eri teknologiat. Esimerkiksi kaupoissa VLC teknologia toimii hyvin, kun valaistus on muutenkin oleellinen asia kaupan tiloissa ja käytännössä kaikilla asiakkailla on olemassa laite joka kykenee VLC teknologiaan. Varastoissa taas valaistus ei välttämättä ole yhtä oleellinen ja niissä seurattavina on yleensä tavarat joihin VLC tekniikan asentaminen on käytännössä mahdotonta, näissä sovelluksissa RFID olisi luultavimmin paras vaihtoehto.

Eri sovelluksiin ja tiloihin sopii siis eri teknologiat. Mitkä teknologiat sitten sopivat mihinkin? Tavaroiden paikantamiseen paras vaihtoehto esitellyistä teknologioista on varmasti RFID, sillä seurattavana oleviin tavaroihin voidaan laittaa passiivinen RFID tunniste, jota voidaan seurata, kuten aikaisemmin on esitetty. Passiiviset tunnisteet ovat erittäin hyvä vaihtoehto tähän, sillä ne eivät tarvitse virtalähdettä toimiakseen ja ne ovat halpoja\cite{F}. Ihmisten paikannukseen paras vaihtoehto on ihmisten puhelimien paikantaminen, sillä käytännössä kaikilla on sellainen aina mukana. Tähän soveltuvia teknologioita edellä mainituista ovat WLAN, Bluetooth ja magneettianturi. Liikeanturitkin ovat periaatteessa toimiva vaihtoehto puhelimiinkin, mutta paikannuksen tarkkuus olisi niin heikko, ettei siitä olisi mitään hyötyä. WLAN on hyvin yleinen ja WLAN tukiasemia löytyy käytännössä joka puolelta. WLAN teknologian infrastruktuuri ei vaatisi suuria muutoksia jo olemassa oleviin systeemeihin, minkä takia WLAN teknologia on hyvinkin houkutteleva vaihtoehto sisätilapaikannukseen. Bluetooth majakoita taas ei ole saman lailla olemassa ja niillä ei olisi muuta tarkoitusta kuin paikannus, toisin kuin WLAN tukiasemat jotka toimivat edelleen langattoman internetyhteyden luomiseen. Toisaalta Applen kiinnostus BLE majakoihin on suuri tekijä ja varmasti edesauttaa niiden yleistymistä. Magneettianturi teknologian vahvuus on sen toiminta ilman erillistä infrastruktuuria jolloin se toimii kaikkialla. Magneettianturi teknologialla on kuitenkin omat ongelmansa virheettömän toimivuuden kanssa kuten aikaisemmin on todettu. 

Kun kaikki tämä laitetaan yhteen, voidaan tehdä arvio tulevaisuuden teknologia tilanteesta. Tulevaisuudessa sisätilapaikannukseen käytetyt teknologiat jakautuvat kahteen, tavaroiden paikannukseen käytetään RFID teknologiaa ja ihmisten paikannukseen tullaan käyttämään WLAN ja magneettianturi teknologioita. WLAN ja magneettiantureitavoidaan käyttää myös yhdessä\cite{atlas}. Tällä tavoin käyttämällä sormenjälkipaikannusta ottamalla huomioon WLAN signaalit sekä maan magneettikenttä saadaan tarpeeksi informaatiota sijainnin luotettavaan paikantamiseen. Nämä vaihtoehdot ovat myös esitetyistä teknologioista kustannuksiltaan ja käytännöllisyydeltään houkuttelevimmat.
%
\subsection{yhteenveto}

Sisätilapaikannukseen on olemassa useita eri teknologioita, näistä tunnetuimmat ovat WLAN, RFID, Bluetooth, VLC, liikeanturit ja magneettianturi. Seuraavassa taulukossa on esitelty nämä teknologiat ja niiden ominaisuuksia. Tämän jälkeen avaamme taulukkoa tarkemmin. Taulukossa on vertailtu jokaisen teknologian tarkkuutta, infrastruktuurin määrää ja herkkyyttä erilaisille häiriöille.
\begin{center}
\begin{tabular}{| l || c | c | c| }
\hline
Teknologia & tarkkuus & infrastruktuuri & häiriö herkkyys\\ \hline\hline
WLAN & hyvä & olemassa & kohtuullinen\\ \hline
RFID & erinomainen & suuri & pieni\\ \hline
VLC & erinomainen & olemassa & suuri\\ \hline
BLE & hyvä & suuri & kohtuulinen\\ \hline
Liikeanturit & heikko & ei tarvitse & erittäin suuri\\ \hline
Magneettianturi & hyvä & ei tarvitse & pieni\\
\hline
\end{tabular}
\end{center}

WLAN teknologialla sormenjälki menetelmää käyttämälllä saadaan hyvä tarkkuus ja tarvittava infrastruktuuri on käytännössä jo olemassa, sillä WLAN tukiasemia löytyy lähes kaikista julkisista tiloista. WLAN teknologia on kohtuullinen häiriö herkkyydeltään, sillä monet asiat, jopa ihmiset muuttavat signaalien vahvuuksia.

RFID teknologialla saadaan erittäin suuri tarkkuus jos tunniste verkko on tarpeeksi tiheä. Teknologia vaatii siis suurta infrastruktuuria, mutta sillä saadaan hyviä tuloksia pienillä häiriöillä, ja se onkin omiaan muun muassa varasto sovelluksissa tavaroiden paikantamiseen.

VLC eli näkyvän valon kommunikaatiolla saadaan erittäin hyvä tarkkuus, eikä se vaadi sen ihmeellisempää infrastruktuuria. kuin led valaisimet. Ongelmaksi muodostuu vastaanottimen asennon tärkeys. Vastaanottimen täytyy olla melko tarkasti vaakatasossa tarkkuuden saavuttamiseksi.

BLE eli Bluetooth Low Energy majakoilla saadaan karkea sijainti, mutta käyttämällä sormenjälki menetelmää, päästään hyvään tarkkuuteen. Tarkkuuden saavuttamiseksi BLE majakoita pitäisi olla asennettuna melko tiheäksi verkoksi. BLE teknologia kärsii samoista ongelmista WLAN teknologian kanssa signaalien vaimenemisen osalta.

Liikeantureilla sijainnin tarkkuus on parhaimmillaankin heikko, sillä pienetkin virheet kumuloituvat nopeasti suuriksi. Etuna on toiminta ilman erillistä infrastruktuuria.

Magneettiantureilla käyttämällä sormenjälki menetelmää maan magneettikentän voimakkuudesta ja suunnasta päästään hyvään tarkkuuteen ilman erillistä infrastruktuuria, sillä maan magneettikenttä löytyy joka puolelta. Magneettikenttä on myös melko muuttumaton joten häiriöt jäävät pieniksi. 

Tulevaisuudessa sisätilapaikannuksen yleistyessä on odotettavissa, että RFID vallitsee tavaroiden paikannus sovelluksissa, esimerkiksi varastoissa. Ihmisten paikannuksessa todennäköistä on, että WLAN ja magneettianturi teknologiat yleistyvät. Näillä kahdella teknologialla on hyvät edellytykset tarkkuuden, häiriöiden ja vaadittavan infrastruktuurin osalta lisäksi yhdessä näillä teknologioilla voidaan täydentää toisen heikkouksia toisen vahvuuksilla.

Tässä luvussa on käsitelty teknologioita ja tekniikoita sisätilapaikannuksen toteuttamiseksi. Tästä on hyvä jatkaa seuraavassa luvussa sillä, mitä näillä voidaan sitten tehdä. Käsittelemme siis sisätilapaikannuksen sovelluksia.

%summa summarum\


