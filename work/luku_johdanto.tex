\section{Johdanto}

%johdatus
%Tämä kandidaatintyö käsittelee sisätilapaikannuksen teknologioita ja sovelluksia.
Reaaliaikaisesta sijaintitiedosta on tullut tärkeä osa monissa sovelluksissa ja järjestelmissä\cite{D}. GPS tarjoaa ratkaisun tähän ulkotiloissa, mutta sisätiloissa GPS:n riittävä tarkkuus ei ole käytännössä mahdollista, sillä sisätiloissa on paljon esteitä kuten seinät, katto ja huonekalut. GPS lähettimen ja vastaanottimen välillä pitää olla näköyhteys, jotta se toimisi luotettavasti.\cite{D,B}
%mitä tutkii
%miksi
Sisätilapaikannuksen on esitetty olevan seuraava askel langattomien järjestelmien ja sovelluksien kehityksessä\cite{C}. Sillä on lisäksi monia käytännön sovelluksia, kuten ihmisten kulun seuranta ja tavaroiden sijainnin selvittäminen varastoissa\cite{A}. 
%tutkimusoglemat
Työn tarkoituksena on tutkia erilaisten teknologioiden soveltuvuutta sisätilapaikannukseen, vertailla niiden vahvuuksia ja heikkouksia sekä tutkia sisätilapaikannuksen sovelluksia. Työssä tehdään myös katsaus näiden tulevaisuuteen.
%tavoitteet
Työn tavoitteena on selvittää mikä esitetyistä teknologioista olisi vallitseva tulevaisuudessa, vai onko valloilla useita eri teknologioita sisätilapaikannukseen. Tavoitteena on myös arvioida minkälainen sovellus tulee olemaan ponnahduslautana sisätilapaikannuksen yleistymiseen.
%menetelmä ainesto
%tulokset?
%työn sisältö ja rakenne
Työn rakenteeseen kuuluu johdantoluku, kolme käsittelylukua sekä yhteenvetoluku. Jokaisessa käsittelyluvussa toistuu sama rakenne. Ensin johdanto aiheeseen ja luvun rakenteen selostus, sitten aiheen käsittely, sitten katsaus tulevaisuuteen aihepiirin osalta ja lopuksi yhteenveto aiheesta.  
Työn ensimmäisessä luvussa käsitellään erilaiset teknologiat sisätilapaikannuksen toteuttamiseksi. Teknologioita vertaillaan soveltuvin osin toisiinsa ja lopussa tehdään katsaus niiden tulevaisuuteen. Seuraavassa luvussa käsitellään sisätilapaikannuksen sovelluksia. Ensin yleisesti ja sitten painopiste siirtyy kaupan alan sovelluksiin. Kolmannessa luvussa käsitellään sisätilapaikannuksen täsmämarkkinointi sovelluksia. Viimeisenä lukuna on yhteenveto.
 %TODO miksi kiinnostavaa?, market size