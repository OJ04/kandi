\section{Johdanto}

%johdatus
%Tämä kandidaatintyö käsittelee sisätilapaikannuksen teknologioita ja sovelluksia.
Reaaliaikaisesta sijaintitiedosta on tullut tärkeä osa monissa sovelluksissa ja järjestelmissä\cite{D}. Erityisesti mainostajille reaaliaikaisesta sijaintitiedosta on tullut tärkeä väline\cite{geo}. GPS tarjoaa ratkaisun tähän ulkotiloissa, mutta sisätiloissa GPS:n riittävä tarkkuus ei ole käytännössä mahdollista, sillä sisätiloissa on useita esteitä kuten seinät, katto ja huonekalut jotka häiritsevät GPS signaalin kulkua. GPS lähettimen ja vastaanottimen välillä pitää olla näköyhteys, jotta luotettava sijaintitieto on mahdollista luoda.\cite{D,B} 
%mitä tutkii
%miksi
Sisätilapaikannuksen on esitetty myös olevan seuraava askel langattomien järjestelmien ja sovelluksien kehityksessä\cite{C}. Sillä on lisäksi monia käytännön sovelluksia, kuten ihmisten kulun seuranta ja tavaroiden sijainnin selvittäminen varastoissa\cite{A}.
Sisätilapaikannuksen nousua tukee myös arviot sen markkinaosuudesta. Sisätilapaikannuksen markkinaosuus on arvioidussa 36,5 prosentin vuosittaisessa nousussa, nousten vuoden 2014 935,05 miljoonan dollarin osuudesta vuoden 2019 arvioituun 4,42 miljardin dollarin osuuteen\cite{reuters}.
%tutkimusoglemat

Työn tarkoituksena on tutkia erilaisten teknologioiden soveltuvuutta sisätilapaikannukseen, vertailla niiden vahvuuksia ja heikkouksia sekä tutkia sisätilapaikannuksen sovelluksia. Työssä tehdään myös katsaus näiden tulevaisuuteen ja tavoitteena on selvittää mikä esitetyistä teknologioista olisi vallitseva tulevaisuudessa, vai onko valloilla useita eri teknologioita yhtäaikaa. Tavoitteena on myös arvioida minkälainen sovellus voisi tulla toimimaan ponnahduslautana sisätilapaikannuksen yleistymiseen.

%menetelmä ainesto
%tulokset?
%työn sisältö ja rakenne
Työn rakenteeseen kuuluu johdantoluku, kolme käsittelylukua sekä yhteenvetoluku. Jokaisessa käsittelyluvussa toistuu sama rakenne. Ensin johdanto aiheeseen ja luvun rakenteen selostus, sitten aiheen käsittely, lopuksi yhteenveto aiheesta ja katsaus tulevaisuuteen.  
Työn ensimmäisessä luvussa käsitellään erilaiset teknologiat sisätilapaikannuksen toteuttamiseksi. Teknologioita vertaillaan soveltuvin osin toisiinsa ja sen perusteella tehdään arvio niiden tulevaisuudesta. Seuraavassa luvussa käsitellään sisätilapaikannuksen sovelluksia. Ensin yleisesti ja sitten painopiste siirtyy kaupan alan sovelluksiin. Kolmannessa luvussa käsitellään sisätilapaikannuksen täsmämarkkinointi sovelluksia. Viimeisenä lukuna on yhteenveto.
 %TODO miksi kiinnostavaa?, market size