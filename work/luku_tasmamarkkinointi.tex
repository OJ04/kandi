\section{Sisätilapaikannus täsmämarkkinoinnissa}
Tässä luvussa käsitellään sisätilapaikannusta täsmämarkkinoinnnissa. Aluksi käsitellään täsmämarkkinointia yleisesti ja mitä keinoja siihen tähän mennessä on ollut käytössä. Seuraavana vaatimukset, eli mitä teknologioilta vaaditaan täsmämarkkinoinnin onnistumiseksi. Seuraavana sisätilapaikannuksen hyödyt täsmämarkkinoinnissa ja lopuksi aiheen tulevaisuus.
\subsection{Täsmämarkkinointi}
Täsmämarkkinointi on mainonnan kohdentamista henkilökohtaiselle tasolle. Jokaiselle täsmämarkkinoinnin kohteena olevalle räätälöidään omat mainokset. Täsmämarkkinoinnin määrä kasvaa perinteisten mainoksien tehon heiketessä mainoksien ja mainonta kanavien lisääntyessä\cite{tasma}. Kilpailu kuluttajan huomiosta kiihtyy ja markkinoinnin vaikutuksen lisäämiseksi kehitellään uusia menetelmiä.
Täsmämarkkinoinnilla voidaan mainostaa ja olla mainostamatta tietynlaisia tuotteita tietynlaisille ihmisille. Esimerkiksi kissan omistajalle ei kannata mainostaa koiran ruokaa ja toisin päin, tämä mainostila kannattaa käyttää sellaisiin mainoksiin, jotka todennäköisesti johtavat tuotteen ostoon. Sisätilapaikannus tuo tilanteeseen uuden aseen, nimittäin markkinoinnin tapahtumisen oikeaan aikaan ja oikeassa paikassa. 
\subsection{Vaatimukset paikannusteknologioille}
%vaaditaan sopiva tarkkuus
\subsection{Sisätilapaikannuksen hyödyt markkinoinnissa}
%asiakas saa hyödyllisiä tarjouksia, mainostaja tuottoa
\subsection{Sisätilapaikannuksen ja markkinoinnin tulevaisuus}
\subsection{Yhteenveto}
