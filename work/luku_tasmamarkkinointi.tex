\section{Sisätilapaikannus täsmämarkkinoinnissa}
%miksi kiinnostavaa
Tässä luvussa käsitellään sisätilapaikannusta täsmämarkkinoinnnissa. Täsmämarkkinointi on eritysen kiinnostava aihe sisätilapaikannuksessa kiinnostavien sovelluksiensa ja niiden moninaisten hyötyjen puolesta mutta myös niiden tuloksien puolesta. BIA/Kelsey\cite{press} lehdistötiedotteessa on esitetty, että sijaintiin perustuvalla markkinoinnilla tuottivat 6.8 miljardia dollaria Yhdysvalloissa vuonna 2015 ja että ne tulevat nousemaan 18.2 miljardiin vuoteen 2019 mennessä.
Luvussa käsitellään aluksi täsmämarkkinointia yleisesti ja mitä keinoja siihen tähän mennessä on ollut käytössä. Seuraavana vaatimukset, eli mitä teknologioilta vaaditaan täsmämarkkinoinnin onnistumiseksi. Seuraavana sisätilapaikannuksen hyödyt täsmämarkkinoinnissa ja lopuksi aiheen tulevaisuus.
\subsection{Täsmämarkkinointi}
Täsmämarkkinointi on mainonnan kohdentamista henkilökohtaiselle tasolle. Jokaiselle täsmämarkkinoinnin kohteena olevalle räätälöidään omat mainokset. Täsmämarkkinoinnin määrä kasvaa perinteisten mainoksien tehon heiketessä mainoksien ja mainonta kanavien lisääntyessä\cite{tasma}. Kilpailu kuluttajan huomiosta kiihtyy ja markkinoinnin vaikutuksen lisäämiseksi kehitellään uusia menetelmiä.

Täsmämarkkinoinnilla voidaan mainostaa ja olla mainostamatta tietynlaisia tuotteita tietynlaisille ihmisille. Esimerkiksi kissan omistajalle ei kannata mainostaa koiran ruokaa ja toisin päin\cite{tasma}, tämä mainostila kannattaa käyttää sellaisiin mainoksiin, jotka todennäköisesti johtavat tuotteen ostoon. Sisätilapaikannus tuo tilanteeseen uuden aseen, nimittäin markkinoinnin tapahtumisen oikeaan aikaan ja oikeassa paikassa.

Tällä hetkellä täsmämarkkinointi on vahvimmin käytössä internetissä, jossa tiedonkeruu on helppoa. Asiakkaista voidaan kerätä tietoja kuten millä sivuilla on käyty, mitä tuotteita on katsottu ja kuinka kauan tuotetta on katsottu. Näiden tietojen perusteella voidaan arvioida mitkä tuotteet voisivat kiinnnostaa kyseistä asiakasta. Asiakkaita voidaan seurata IP osoitteen tai käyttäjätunnuksen avulla.
Toinen tällä hetkellä käytössä oleva menetelmä on kohdentaa mainostus erilaisille ihmisryhmille heitä kiinnostavalla tavalla. Esimerkiksi opiskelijoille voidaan tarjota opiskelija alennuksia erilaisista tuotteista ja palveluista. Esimerkiksi Siwa tarjoaa opiskelijoille alennuksia kaikista ruokaostoksissta\cite{siwa} ja näin houkuttelee opiskelijoita käymään Siwassa kilpailijoidensa sijasta. 

%jotain lajittelua taas ehkä
\subsection{Sovellukset ja vaatimukset}
%vaaditaan sopiva tarkkuus
Monissa sisätilapaikannuksen täsmämarkkinointi sovelluksisa riittää tieto, onko asiakas kaupassa tai kauppakeskuksessa vai ei. Tähän riittää heikkokin tarkkuus. Kaikki luvussa 1 esitetyt teknologiat käyvät tähän oikein hyvin lukuunottamatta liikeantureita. Liikeanturit olisivat tässä tapauksessa lähes välttämättömästi puhelimen liikeantureita joilla ei saada tarpeeksi hyvää tarkkuutta edes tälläiseen sovellukseen. Tälläisiä sovelluksia voisi olla esimerkiksi ravintoloiden jakelemat tarjoukset tai uuden liikkeen avajais ilmoitukset.

Alaluvussa 3.2 puhuttiin sovelluksesta jolla kaverit voivat löytää toisensa kauppakeskuksessa. Tähän voidaan yhdistää täsmömarkkinointi siten, että esimerkiksi neljän hengen kaveriporukalle lähetetään tarjous juuri neljälle hengelle viereisestä kahvilasta. Vaikka kaveriporukan ei aikaisemmin ollut tarkoitus jäädä kahville, tämänlaiset räätälöidyt tarjoukset houkuttelevat heitä jäämään. Tämä ei ole mahdollista ilman sisätilapaikannusta, sillä toimiakseen hyvin, se vaatii tarjouksen tulemista oikeaan aikaan oikeassa paikassa. Kotiin tuleva munkki kahvi tarjous neljälle hengelle ei tehoa läheskään yhtä hyvin, kuin kahvilan vieressä tapaavalle kaveri porukalle. Tähän tarvitaan jo kymmenen metrin luokkaa olevaa tarkkuutta.

Suurinta tarkkuutta vaaditaan täsmämarkkinointi sovelluksissa joissa markkinoidaan yksittäisiä tuotteita. Tämän tyyppisissä sovelluksissa tarkkuuden tulee olla korkea, sillä mitä lähempänä asiakas on tarjouksen tuotteita, sitä todennäköisemmin hän niihin tarttuu. Siksi tarkkuuden tulee olla korkeintaan metrin luokkaa. Kauppojen hyllyjen välissä suurempi virhe voi siirtää sijainnin viereiselle käytävälle, mikä suorana etäisyytenä on lähellä, mutta sinne päästäkseen täytyy usein kiertää kaukaa.

Suurimmassa osassa tapauksista paikannukseen käytetään puhelimen antureita ja muita ominaisuksia. Tämä tuo omat haasteet sillä paikannuksen onnistumiseksi puhelimissa täytyy olla paikannus ominaisuus kytkettynä päälle, eikä asiakkaat välttämättä aina muista, jaksa tai halua kytkeä paikannusta päälle. Tähän ratkaisuksi voisi toimia myös luvussa 3.2 käsitelty tekniikka jossa seurataankin kaupan ostoskärryjä ja koreja. Tässä ongelmana on kuitenkin asiakkaiden tunnistaminen. Kärryn sijainnin määrittäminen onnistuu helposti mutta kärryn kuljettajan määrittäminen on mahdotonta ilman minkäänlaista tunnistautumista. Täsmämarkkinoinnin onnistumiseksi tarvitaan nimenomaan tieto \textit{kuka} kulkee milloin missäkin, jotta voidaan tarjota personoitua sisältöä. Siksi kannattaakin kannustaa asiakkaita kytkemään paikannus päälle. Tämä onkin täsmämarkkinoinnin vahvuus, sillä sallimalla täsmämarkkinointi asiakas hyötyy siitä itse. Käsittelemme hyötyjä asiakkaalle ja mainostajalle seuraavassa alaluvussa.

\subsection{Sisätilapaikannuksen hyödyt markkinoinnissa}
%asiakas saa hyödyllisiä tarjouksia, mainostaja tuottoa'
Täsmämarkkinointi on hyödyksi kaikille osapuolille, kuluttajat saavat tietoa juuri heitä kiinnostavista tuotteista ja mainostajat parempaa tuottoa kuin perinteisestä mainonnasta.



%kuluttajalle
Kuluttajat saavat tietoa heitä kiinnostavista tuotteista, jolloin he saavat mitä haluavat ja ovat tyytyväsiä. Lisäksi kuluttajien ei tarvitse nähdä mainoksia jotka eivät kiinnosta. Tämä vähentää ärsyyntymistä mainoksiin, mikä parantaa mielikuvaa mainostajasta ja mainoksista. Kuitenkin suurin hyöty kuluttajan näkökulmasta ovat varmasti tarjoukset ja etukupongit joilla heitä huokutellaan ostamaan jokin tuote tai käyttämään jotain palvelua. Tarjouksien kohdistuminen juuri heitä kiinnostaviin tuotteisiin tekee niistä erityisen kiinnostavia ja hyödyllisiä.
 

%kaupalle/mainostajalle
Kaupat keräävät nyt jo tietoa siitä mitä kukin asiakas on ostanut. Sisätilapaikannuksen avulla nämä tiedot saadaan hyötykäyttöön reaaliaikaisesti jo kaupassa.
Mainostajan ei kannata tuhlata aikaa ja resursseja näyttämällä kuluttajille sellaisia mainoksia jotka eivät heitä varmasti kiinnosta, sillä ne eivät johda tuotteen ostoon. Mainokset, jotka kiinnostavat kuluttajia taas johtavat paljon todennäköisemmin tuotteen ostoon. Tutkimuksessa\cite{target} onkin todettu että täsmämarkkinoinnilla saadaan keskimäärin 2.7 kertaiset tuotot jokaista mainosta kohden.
Reaaliajassa reagoiva markkinointi auttaa tasaamaan kuormitusta ja optimoi toiminnan volyymiä\cite{tasma}. Jos ravintolassa on esimerkiksi hiljaista, kannattaa sen mainostaa ruokalistaansa lähistöllä oleville potentiaalisille asiakkaille. Sisätilapaikannuksella voidaan seuloa asiakkaista sellaiset jotka ovat olleet liikkeellä jo pitkään ja ovat todennäköisesti nälkäisiä. Myös allergiat on helppo otta huomioon, jolloin tietyt ruoat eivät näy listassa lainkaan\cite{tasma}. Mainontaa voidaan vähentää kun ravintola on täynnä, jotta vältytään jonottamiseen ärsyyntyneiltä asiakkailta.
Kauppa voi houkutella ohikäveleviä asiakkaita sisään vaihtamalla ikkunassa näkyviä tarjouksia sen mukaan mikä ohikulkijaa kiinnostaa ja näin parantaa omaa tuottoaan.

\subsection{Yhteenveto ja tulevaisuus}

%räätälöinti number uno!!
