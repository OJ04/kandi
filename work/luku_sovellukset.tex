\section{Sisätilapaikannuksen sovellukset}
Tässä luvussa käsitellään sisätilapaikannuksen sovelluksia. Alkuun käsitellään sovelluksia yleisesti ja sen jälkeen siirrämme painopisteen kaupan alan sovelluksiin. Lopuksi teemme katsauksen tulevaisuuteen sisätilapaikannuksen sovelluksien osalta ja esitäme arvioita minkä tyyppisellä sovelluksella tehdään läpimurto ja sisätilapaikannuksen sovellukset alkavat ylestyä.

\subsection{Sovellukset yleisesti}
Sisätilapaikannuksella on monia sovelluksia ja monet nykyiset sovellukset ja järjestelmät hyötyisivät mahdollisuudesta tietää itsensä tai käyttäjänsä sijainti.Yleisesti sisätilapaikannus yhdistetään vain sisätila navigointiin mutta sisätilapaikannus tarjoaa paljon muitakin mahdollisuuksia. Seuraavassa esimerkkejä sisätilapiaikannuksen sovelluksista. 

\paragraph{Varastot} Tavaroiden sijainnin ja lukumäärän selvittäminen. Tällä tavalla voidaan pitää pienempiä varastoja kun tidetään varaston saldot tarkalleen reaaliaikaisesti. Voidaan käyttää myös optimaalisen reitin löytämiseen useaa tavaraa haettaessa. Varaston automatisointi on myös mahdollista sisätilapaikannuksen avulla.
\paragraph{Sairaalat} Äkilllisissä tilanteissa lääkärin tai muun sairaala henkilökunnan löytäminen nopeasti\cite{ibeacon} tai automaattisesti lähimmän hoitajan hälyttäminen paikalle. Myös potilaiden paikannus on tärkeä sovellus varsinkin vakavasti sairaiden potilaiden kohdalla.
\paragraph{Robotit} Navigointi kyky sisätiloissa. Lisää työturvallisuutta teollisuuden koneiden läheisyydessä ja parantaa sisätiloissa liikkuvien koneiden toimintakykyä.\cite{lips}
\paragraph{Kampus} Luentosalin löytäminen, hyödyllinen etenkin uusille opiskelijoille ja vieraileville luennoitsijoille \cite{campus}.
\paragraph{Museo} Edessä olevan taideteoksen esittely esimerkiksi puhelimeen. Mahdollisesti käyttäjää kiinnostavien teoksien luokse ohjaaminen. Museolle arvokasta tietoa asiakkaiden kulkureitestä palveluiden parantamiseksi.
\paragraph{Parkkihallit} Vapaan paikan löytäminen tai oman auton löytäminen\cite{lips}.


\subsection{Kaupan alan sovellukset}
navigointi
heat map
mainonta
\subsection{Sovellukset tulevaisuudessa}
killeri appi
\subsection{Yhteenveto}

